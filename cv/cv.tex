\documentclass{resume} % Use the custom resume.cls style
\usepackage[dvipsnames]{xcolor}
\usepackage{hyperref}
\usepackage{tabularx}
\usepackage{enumitem}
\usepackage[left=0.4 in,top=0.3 in,right=0.4 in,bottom=0.3in]{geometry} % Document margins
\newcommand{\tab}[1]{\hspace{.2667\textwidth}\rlap{#1}}
\newcommand{\itab}[1]{\hspace{0em}\rlap{#1}}
\hypersetup{
    colorlinks=true,
    linkcolor=blue,
    filecolor=magenta,
    urlcolor=blue,
}
% \usepackage{array}
\name{Anning Gao} % Your name
% \address{C-101, Dreamland Apartment, Backbone Park, B/H Balaji Hall, Rajkot-360004, Gujarat} % Your address
%\address{123 Pleasant Lane \\ City, State 12345} % Your secondary addess (optional)
% \address{+86 19800363861 \\ gao.2526@osu.edu} % Your phone number and email
\address{gao.2526@osu.edu \\ \href{https://anninggao.github.io}{https://anninggao.github.io}}
\definecolor{TsinghuaPurple}{cmyk}{0.58,0.90,0,0}
\renewenvironment{rSection}[1]{
\sectionskip
\textcolor{TsinghuaPurple}{\MakeUppercase{#1}}
\sectionlineskip
\hrule
\begin{list}{}{
%\setlength{\leftmargin}{1.5em}
\setlength{\leftmargin}{0em}
}
\item[]
}{
\end{list}
}


\begin{document}

\begin{rSection}{Research INTERESTS}

\begin{enumerate}
    \item Observational cosmology: redshift survey / lensing / CMB, BAO, Lyman-$\alpha$ forest, non-standard statistics 
    \item Medium physics: IGM \& reionization, CGM \& galaxy ecosystem, multiband observation (UV / radio / ...)
    \item Galaxy physics: co-evolution with dark matter halo \& SMBH, star formation activity, feedback from AGN \& SN 
    \item Statistics \& machine learning: Bayesian methods, neutral network applications on observations \& numerical simulations 
    % \item High energy transients: 
\end{enumerate}

\end{rSection}

\begin{rSection}{Education}
    \begin{rSubsection}{Tsinghua University} {2020 - 2025}{B.S. in Physics \hfill \textup{Beijing, China}}{}
        \item Thesis: Cosmology Constraints from Cosmic Void Size Function
    \end{rSubsection}

    \begin{rSubsection}{The Ohio State University} {2025 - Present}{Ph.D. in Astronomy \hfill \textup{Columbus, Ohio, USA}}{}
    \item Advisor: Prof. Yuan-Sen Ting
    \end{rSubsection}

\end{rSection}


\begin{rSection}{Publication}
\textbf{A. Gao}, J. X. Prochaska, et al., "Measuring the Mean Free Path of HI Ionizing Photons at $3.2\leq z\leq4.6$ using DESI Y1 Quasars", 2025, Astrophysical Journal Letters 981 L27, \href{https://arxiv.org/abs/2411.15838}{arXiv:2411.15838}.
\end{rSection}


\begin{rSection}{RESEARCH EXPERIENCE}

    \begin{rSubsection}{Measuring the Mean Free Path (MFP) of HI Ionizing Photons using DESI Data} {Jul. 2023 - Sep. 2024} {Supervisor: Prof. Jason X. Prochaska (UC Santa Cruz)}

    \item Constructed the dataset from DESI Y1 QSO spectra , which is $\sim10$ times larger than previous datasets.
    \item Wrote the \href{https://github.com/AnningGao/MeanFreePath}{package for MFP calculation} and measured the MFP at $3.2<z<4.6$ with unprecedented precision
    \item Validated our result on the selection of quasar SED, the treatment of Lyman series opacity and the MFP definition.
    \item Confirmed that MFP evolves much slower than previous estimates with high confidence.
    \item Our result presents challenges to the current IGM absorber models and gives new constraints on the reionization history.
        
    \end{rSubsection}

    \begin{rSubsection}{Constraining the Cosmology with Void Size Function based on Delaunay Triangulation}{Mar. 2024 - Present}{Supervisor: Prof. Cheng Zhao (Tsinghua University)}

    \item Constructed the void catalog from ~40,000 Quijote halo catalogs with different cosmological parameters.
    \item Trained an AI-based emulator to predict the void size function under different cosmologies.
    \item Tested the emulator on Quijote and performed MCMC fitting for the cosmological parameters.
    \item {\bf Future:} Improve the method to remove the severe systematics and apply the method to the BOSS galaxy catalog.

    \end{rSubsection}

    \begin{rSubsection}{High-redshift CGM OVI Absorption Analysis using DESI Data} {Feb. 2023 - Feb. 2024}{Supervisor: Prof. Zheng Cai and Dr. Siwei Zou (Tsinghua University)}
        
    % \item Use multiple tools like Pyraf, EAZY and SExtractor to do photometric measurement on JWST images.
    \item Fitted the continuum of $\sim15,000$ $z>2.5$ DESI QSO spectra with a newly proposed unsupervised machine learning method.
    \item Visually inspected $\sim2000$ spectra to find DLA-related OVI absorption features, found $\sim30$ OVI absorption candidates and fitted the Voigt profile to extract column densities and the Doppler parameter.
    \item Statistically analysed the OVI absorption sample to find the properties of high-$z$ warm-phase CGM.
    \item Performed simulations using Cloudy to get ionization parameters and hydrogen number densities that cannot be directly observed.
    % \item Wrote the preprocess code for DESI spectrum with functions like dealing with the overlapping area, clipping the abnormal point, rebining to the same restframe grid, masking the DLA, etc.
    % \item Provide training samples for a ML model QFA to better fit the continuum of the quasar spectrum with Lyman alpha forest.
    % \item {\bf Future:} Predict the mass of the central black hole of AGN based on reverberation mapping and explore which emission line is the most useful to the prediction.
    \end{rSubsection}

    %------------------------------------------------

    % \begin{rSubsection}{Study of the Transient AT2022jrp} {Sep. 2022 - Jan. 2023}{Supervisor: Prof. Sharon Xuesong Wang (Tsinghua University)}

    % \item Selected the target, wrote the observing proposal with our group and conducted the observation at Xinglong Observatory.
    % \item Checked the periodic characteristics of the transient's light curve together with AAVSO's data.
    % \item Found the correlation between the transient's magnitude and H$\alpha$ line EW using Pearson coefficient.
    % % \item {\bf Future:} Model the H$\alpha$ emission from radiative transfer.
    % % \item Paper in preparation.
    % \end{rSubsection}

    % \begin{rSubsection}{Search for new multi-quark states in CMS experiments} {Nov. 2021 - Aug. 2022}{Supervisor: Prof. Zhen Hu (Tsinghua University)}{}

    % \item Used ROOT to analyse Trees, plotted histograms and fitted the data.
    % \item Used CMSSW and CRAB to submit jobs to the CERN computing cluster and conduct Monte Carlo simulations.
    % \item Generated multiple Monte Carlo datasets in the cross section measurement of a new tetraquark state made of charm quarks.

    % \end{rSubsection}
\end{rSection}

\newpage

\begin{rSection}{scientific talks}
    \begin{rSubsection}{Measuring the Mean Free Path of HI Ionizing Photons using DESI Data}{Jul. 2024}{}{}
    \item Oral Presentation, 2024 DESI Summer Meeting, Marseille, France
    \end{rSubsection}

    \begin{rSubsection}{Introduction to Normalizing Flow}{May. 2024}{}{}
    \item Machine Learning Seminar, Department of Astronomy, Tsinghua University
    \end{rSubsection}

    \begin{rSubsection}{Measuring the Mean Free Path of HI Ionizing Photons using DESI Data}{Sep. 2023}{}{}
    \item Oral Presentation, 2023 PKU Undergraduate Astronomy Symposium, Beijing, China
    \end{rSubsection}
\end{rSection}


\begin{rSection}{Skills}
    \noindent\textbf{Skilled in:} Python, C/C++, Mathematica, Shell/Bash, \LaTeX, Git.

    \vspace{0.5em}
    
    \noindent\textbf{Experienced with:}
    \begin{itemize}[nosep]
        \item Packages \& softwares for IGM and absorption analysis: Cloudy, VoigtFit, Pyigm, Linetools
        \item Manipulating catalogs, analyzing dataset and visualization
        \item Machine learning and building neural network models: normalizing flow, operator learning, ...
        \item Statistical analysis: MCMC sampling, Bayesian inference, nonparametric methods, ...
    \end{itemize}

\end{rSection}

% \begin{rSection}{Selected Coursework}
%     \begin{tabular}{rp{0.4\textwidth}}
%         \begin{minipage}{0.5\linewidth}
%             \centering {\bf Physics \& Astronomy}

%             \begin{tabular}{m{0.7\linewidth} m{0.15\linewidth}}
%                 \item Quantum Mechanics (1) & A-\\
%                 \item Analytical Mechanics & A\\
%                 \item Nuclear and Particle Physics & A\\
%                 \item Observational Astronomy & A\\
%                 \item Galaxies and the Universe & A\\
%                 \item Stars and Planets & A\\
%                 \item Galatic Physics & A-\\
%         \end{tabular} \end{minipage}
%         \begin{minipage}{0.5\linewidth}
%             \centering {\bf Mathematics \& CS}

%             \begin{tabular}{m{0.7\linewidth} m{0.15\linewidth}}
%                 \item Complex Analysis & A\\
%                 \item Basic Topology & A-\\
%                 \item Group Theory & A-\\
%                 \item Probability Theory & A\\
%                 \item Multidimensional Statistical Inference & A\\
%                 \item Machine Learning & A\\
%                 \item Deep Learning & A-\\
%             \end{tabular}
%         \end{minipage}
%     \end{tabular}
% \end{rSection}

\begin{rSection}{Scholarships and Awards} \itemsep -2pt
{Scholarship of the National Astronomical Observatory of China}\hfill {2024\;\;\;\;\;\;\;\;\;} \\
{Lin-bridge Scholarship, Department of Astronomy, Peking University}\hfill {2023\;\;\;\;\;\;\;\;\;}\\
{Chi-Sun Yeh Scholarship, Member of the Tsinghua Xuetang Talents Program}\hfill {2021, 2022, 2023} \\
{Scholarship for Academic Excellence, Department of Physics, Tsinghua University}\hfill {2021, 2022, 2024}\\
{University Fellowship, The Ohio State University}\hfill {2025\;\;\;\;\;\;\;\;\;}\\
% {Five-star Volunteer, Tsinghua University}\hfill {2022\;\;\;\;\;\;\;\;\;}\\
\end{rSection}


% \begin{rSection}{References}
%     \begin{tabularx}{\textwidth}{ 
%          >{\raggedright\arraybackslash}X 
%          >{\centering\arraybackslash}X 
%          >{\centering\arraybackslash}X  }
%         \textbf{Prof. Jason X. Prochaska} & xavier@ucolick.org & University of California, Santa Cruz\\
%         \textbf{Prof. Zheng Cai} & zcai@mail.tsinghua.edu.cn & Tsinghua University\\
%         \textbf{Prof. Cheng Zhao} & czhao@mail.tsinghua.edu.cn & Tsinghua University\\
%     \end{tabularx}

% \end{rSection}
\end{document}
